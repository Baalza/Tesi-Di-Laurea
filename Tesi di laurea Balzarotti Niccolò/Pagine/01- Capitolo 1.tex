\mainmatter

\chapter{Storia Del Cloud Computing}

\section{La nascita e l'evoluzione del Cloud Compunting}
A porre le basi del cloud computing negli anni ‘60 è stato il professor John McCarthy, scienziato informatico noto anche per aver coniato il termine Intelligenza Artificiale.
In un celebre discorso tenuto al MIT (Massachusetts Institute of Technology), John McCarthy ha introdotto l’idea di Time Sharing, ossia di un computer in grado di supportare la presenza simultanea di più utenti.


Sempre nel 1961 fu McCarthy a introdurre il concetto di Utility computing, un modello di fornitura di servizi on demand per i clienti con tariffe a consumo.
Il cloud computing che conosciamo oggi però non nasce solo dalle sue idee, ma anche da  quelle di Joseph Carl Robnett Lickelider, il primo a parlare di “\textit{rete intergalattica di computer}”.\cite{historyOfCloud}

\subsection{La “rete intergalattica di computer” di Lickelider}
In molti fanno risalire la nascita del cloud computing proprio a Joseph Licklider che, nel 1969, lavorava allo sviluppo del famoso ARPANET (Advanced Research Projects Agency Network), una rete commissionata dal Dipartimento della Difesa degli Stati Uniti per  scopi militari. Questa rete consentiva uno scambio di informazioni veloce, sicuro ed efficiente rendendolo sostanzialmente il network progenitore di internet. 
Nella visione di Licklider tutto il mondo poteva essere interconnesso e poteva accedere ai programmi da qualsiasi luogo e in qualsiasi momento portando così alla nascita del \textit{grid computing}, il primo vero antenato del cloud.


L’unione di queste idee e progetti come il  MAC (Multiple Access Computing) del 1965 e il sistema operativo time sharing CP-40/CMS messo a punto da IBM nel 1967, hanno fatto sì che il mondo si aprisse all’idea di un sistema informatico utilizzato da più persone contemporaneamente.\cite{historyOfCloud}

\newpage
\section{La macchina virtuale di IBM}
Le macchine virtuali di IBM negli anni ’70 CP-40/CMS non erano altro che software in grado di eseguire sistemi operativi e applicazioni. In pratica, ognuna di queste macchine possiede dei dispositivi che forniscono le stesse funzionalità dell’hardware fisico e offrono vantaggi in termini di portabilità, gestione e sicurezza, permettendo agli utenti di accedere a un hardware gestito esternamente.
Gli anni ‘70 e ‘80 furono quelli della svolta; in particolare, nel 1976 il funzionamento dei progressi del networking fu mostrato alla regina Elisabetta II, che inviò un’e-mail proprio con ARPANET.\cite{historyOfCloud}

\begin{figure}[!b]
    \centering
    \includegraphics[width=0.75\textwidth]{arpanet-mail.png}
    \caption{ Elisabetta II, la prima sovrana a inviare una e-mail con Arpanet}
    \label{fig:mailArpanet1976}
    \cite{QueenElizabethII}
\end{figure}

\newpage
\section{Il lancio del World Wide Web}
Le tecnologie fondamentali per il funzionamento del cloud si sono consolidate negli anni ’90, semplificate dal lancio del World Wide Web del 1991. 


E’ proprio in questo decennio rivoluzionario per la tecnologia e contraddistinto dalla nascita dei primi e-commerce, dei primi modelli client-server e dei siti web che sviluppano il front-end per gli utenti e il back-end per i tecnici che troviamo la prima menzione ufficiale di cloud computing. Precisamente nel 1996, all’interno di un documento scritto da dirigenti tecnologici di Compaq (azienda statunitense finalizzata alla produzione nel settore dei personal computer), che andò a sostituire l’allora termine popolare “grid computing”.\cite{historyOfCloud}
\vspace{2cm}
\begin{figure}[!h]
    \centering
    \includegraphics[width=0.95\textwidth]{documento-compaq.png}
    \caption{ Prima menzione ufficiale del termine cloud computing}
    \label{fig:menzione1996}
    \cite{Compaq}
\end{figure}

\newpage
\section{La nascita ufficiale del cloud computing}
Nel momento in cui i personal computer sono diventati sempre più accessibili e tutti potevano finalmente connettersi, Salesforce è stata la prima azienda globale a offrire applicazioni in rete in modalità cloud lanciando il Software as a Service alle imprese.

La nascita ufficiale del cloud computing però, coincide con la creazione del primo cloud moderno realizzato da Amazon Web Services nel 2006, offrendo servizi di cloud computing su richiesta a prezzi accessibili e diventando il primo servizio su larga scala per il grande pubblico. A seguire, anche Google ha introdotto la sua Google Cloud Platform e Microsoft nel 2011 ha lanciato la sua prima applicazione, Microsoft Office, accelerando ancor di più la diffusione del cloud computing nel mondo.\cite{historyOfCloud}

