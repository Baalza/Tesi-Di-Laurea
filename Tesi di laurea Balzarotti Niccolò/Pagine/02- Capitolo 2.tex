\chapter{Introduzione Al Cloud Computing}

\section{Cos'è il cloud computing}
Il cloud computing è un modello di erogazione di servizi offerti su richiesta da un fornitore a un utente finale in maniera on-demand e attraverso Internet.
Volendo essere più precisi per addentrarci nell'argomento possiamo utilizzare la definizione di Cloud del NIST (National Institute of Standards and Technology):
“Il cloud computing è un modello che consente un accesso di rete onnipresente, comodo e \textit{on-demand} \footnote{Per on demand o su richiesta, nel campo dell'informatica aziendale, si intende l'accesso alle risorse informatiche tramite internet solo quando necessario, eventualmente pagando le stesse in base all'utilizzo e non in base a un canone fisso, o acquistando una licenza una tantum. \cite{OnDemand} \\} a un pool condiviso
di risorse informatiche configurabili (ad esempio reti, server, storage, applicazioni e servizi) che
possono essere rapidamente fornite e rilasciate con il minimo sforzo di gestione o interazione con il fornitore di servizi.
Questo modello cloud è composto da cinque caratteristiche essenziali, tre modelli di servizio e quattro modelli di distribuzione.”
\cite{NIST}


In un modello cloud possiamo quindi riconoscere tre figure essenziali: \cite{WCloud}
\begin{itemize}
    \item il fornitore di servizi: che offre servizi quali archiviazione,server e applicazioni solitamente secondo un modello \textit{pay per use} (PPU) 
    \footnote{Il termine PPU indica una forma di remunerazione in base alla quale si paga in funzione dell'utilizzo effettivo. \cite{PPU}}.
    \item utente amministratore: sceglie e configura i servizi offerti dal fornitore per l'utente finale.
    \item utente finale: utilizza i servizi forniti. 
\end{itemize}
In determinati casi il cliente amministratore e il cliente finale possono coincidere.


%\myemptypage
%\hspace{1em} % per aggiungere una linea vuota spazio
\newpage
\subsection{Precisazione}
 Molto spesso i servizi cloud vengono erroneamente fraintesi con i servizi offerti da un datacenter, riguardo a questo la direttiva UE 2022/2555 NIS2 definisce: \cite{NIS2}
 \begin{itemize}
    \item Cloud: I servizi di cloud computing dovrebbero comprendere servizi digitali che consentono l'amministrazione su richiesta di un pool scalabile ed elastico di risorse di calcolo condivisibili e l'ampio accesso remoto a quest'ultimo, anche quando tali risorse sono distribuite in varie ubicazioni.
    \item Datacenter: Il termine «servizio di data center» dovrebbe applicarsi alla fornitura di un servizio che comprende strutture, o gruppi di strutture, dedicate a ospitare, interconnettere e far funzionare in modo centralizzato apparecchiature informatiche e di rete che forniscono servizi di conservazione, elaborazione e trasporto di dati insieme a tutti gli impianti e le infrastrutture per la distribuzione dell'energia e il controllo ambientale. 
\end{itemize}

\section{Caratteristiche essenziali dei modelli cloud}
Come definito in precedenza il modello cloud è composto da cinque caratteristiche fondamentali: \cite{NIST}
\cite{Atlassian}
 \begin{itemize}
    \item On-demand self-service : Un utente può richiedere i vari servizi offerti dal fornitore in totale autonomia, senza dover passare per gestori e/o provider.
    \item Broad network access: Tramite hardware fisico distribuito a livello globale, le funzionalità offerte sono disponibili in rete e accessibili ovunque.
    \item Resource Pooling: Le risorse informatiche in un modello cloud vengono raggruppate per servire più utenti
    utilizzando un modello multi-tenant, in questo modo le risorse sono suddivise dinamicamente in base alla domanda dell'utente rendendo così  l'hardware cloud completamente ottimizzato per il massimo utilizzo.
    \item Rapid elasticity: Le infrastrutture cloud possono crescere e ridursi in modo dinamico, consentendo agli utenti di richiedere che le loro risorse di calcolo si adattino automaticamente alle richieste di traffico, avendo così la massima scalabilità.
    \item Measured service: I sistemi cloud controllano e ottimizzano automaticamente l'utilizzo delle risorse che possono essere
    monitorate, controllate e segnalate (tipicamente usate nei PPU - pay per use), garantendo trasparenza sui costi di utilizzo del servizio utilizzato al consumatore.
\end{itemize}
\newpage
\section{Modelli di servizio} \label{ModelliDiServizio}
I servizi cloud sono formati da infrastrutture, piattaforme o software in hosting presso provider esterni e messi a disposizione degli utenti attraverso Internet.
Partendo dal modello “base” chiamato on-premise/on-site in cui è l’utente il proprietario e responsabile di ogni aspetto, dall'hardware alle applicazioni, fino alla scalabilità, esistono tre modelli di servizi cloud principali:
\textit{IaaS} (Infrastructure as a Service \footnote{L’espressione "As a Service" indica che il modello di servizio viene offerto da una terza parte nel cloud e che quindi non è necessario acquistare, gestire o utilizzare hardware, software o strumenti come nel modello on-premise. \cite{GoogleCloud} \\}), \textit{PaaS} (Platform as a Service) e \textit{SaaS}  (Software as a Service). 


Questi termini si riferiscono al modo in cui il cloud viene utilizzato all’interno dell’organizzazione e al grado di gestione di cui si è responsabili. Di seguito andremo a definirne le caratteristiche:
 \cite{RedHat}
 \begin{itemize}
    \item IaaS: Con il modello Infrastructure as a Service, o IaaS, il provider fornisce al cliente un'infrastruttura completa che include server fisici, storage, rete e componenti di virtualizzazione,
    in questo modo l'utente accede a tale infrastruttura, sostanzialmente a noleggio, tramite un'\textit{API} \footnote{API o application Programming Interface, ovvero "interfaccia di programmazione delle applicazioni", in informatica è un insieme di regole e protocolli che consente a diversi programmi software e servizi di comunicare e scambiare dati, caratteristiche o funzionalità tra loro.\cite{IBMAPI} \\} o una dashboard specifica, e gestisce aspetti come il sistema operativo, le app e il \textit{middleware} \footnote{Il middleware è un software che funge da strato intermedio tra le applicazioni e le componenti sottostanti, come ad esempio sistemi operativi, database o hardware, facilitando la comunicazione, l'integrazione e la gestione delle risorse nei sistemi distribuiti. Il suo ruolo primario è quello di nascondere la complessità dell'infrastruttura sottostante, garantendo interoperabilità, scalabilità e sicurezza. \cite{Middleware} \\} senza doversi preoccupare della gestione, della manutenzione o dell'aggiornamento dell'hardware sottostante, dei data center, delle connessioni di rete o degli interventi in caso di guasti.
    Questo modello consente quindi alle aziende di ridurre i costi legati all’infrastruttura fisica e di  concentrarsi sugli aspetti applicativi e di business, lasciando al provider le complessità operative e la continuità del servizio.
    \item SaaS: Con il modello software as a Service, o SaaS, il provider gestisce interamente un'applicazione software che fornisce agli utenti; solitamente, queste applicazioni sono accessibili tramite un browser web che permette di eliminare la necessità di dover installare le applicazioni sui computer dei singoli utenti. L’utente che si connette a queste applicazioni tramite un'API o una dashboard dedicata dovrà preoccuparsi solamente della manutenzione del software.
    \item PaaS: con il modello Platform as a Service, o PaaS, il provider fornisce e si occupa di tutte le risorse hardware e software per lo sviluppo delle applicazioni tramite cloud, mentre l'utente gestisce le app eseguite sulla piattaforma e i dati che esse utilizzano.
    Questi servizi, che inoltre offrono agli utenti una piattaforma cloud condivisa per lo sviluppo e la gestione delle applicazioni (un aspetto importante della metodologia \textit{DevOps} \footnote{DevOps (dalla contrazione inglese di development, "sviluppo", e operations, qui simile a "messa in produzione" o "deployment") è una metodologia di sviluppo del software utilizzata in informatica che punta alla comunicazione, collaborazione e integrazione tra sviluppatori e addetti alle operations, puntando ad aiutare un'organizzazione a sviluppare in modo più rapido ed efficiente prodotti e servizi software.  \cite{DevOps}}), sono pensati soprattutto per sviluppatori e programmatori.
\end{itemize}
\vspace{2cm}
\begin{figure}[!h]
    \centering
    \includegraphics[width=1.0\textwidth]{modelli-cloud.png}
    \caption{ Tabella modelli di servizio cloud}
    \label{fig:modellDiServizioCloud}
    \cite{RedHat}
\end{figure}
\newpage
Oltre a queste tre principali categorie, è possibile trovare anche altri tipi di cloud che usano tecnologie differenti, come ad esempio i \textit{container} \footnote{Container: I container consentono di raggruppare e isolare le applicazioni insieme al relativo ambiente di runtime, che include tutti i file necessari per l'esecuzione. In questo modo è più facile spostare applicazioni tra gli ambienti (sviluppo, test, produzione, ecc.) conservandone tutte le funzionalità.  \cite{Container}
} che hanno avuto sempre più successo nell’architettura a microservizi, portando alla creazione del Modello CaaS (Container as a Service).
Con questo modello il provider fornisce e gestisce tutte le risorse hardware e software per lo sviluppo e il deployment delle applicazioni con i container utilizzati come risorsa principale al posto delle macchine virtuali.
In questo modo sviluppatori e team delle operazioni IT possono sfruttare il modello CaaS per sviluppare, eseguire e gestire le applicazioni senza dover creare e mantenere l'infrastruttura o la piattaforma per l'esecuzione e la gestione dei container preoccupandosi solo della scrittura del codice e della gestione dei dati e delle applicazioni. \cite{GoogleCloud}
\newpage
\section{Modelli di distribuzione}
Dopo aver analizzato le caratteristiche fondamentali del cloud computing, è importante capire come questa infrastruttura viene distribuita e resa disponibile agli utenti.
In base a chi possiede e gestisce le risorse (hardware e  non), il modo in cui sono allocate e gli accessi concessi si denotano diverse tipologie di modelli di distribuzione: \cite{RedHat}
 \begin{itemize}
    \item Cloud pubblico: I cloud pubblici sono ambienti cloud basati su un'infrastruttura IT che non appartiene all'utente finale e solitamente vengono eseguiti off-premise (non fisicamente nei locali del cliente) anche se ad oggi molti provider di cloud pubblici eseguono i propri servizi nei datacenter on-premise dei propri clienti.
    I provider di cloud pubblico più importanti sono Amazon Web Services (AWS), Google CLoud, IBM Cloud, Microsoft Azure e Alibaba Cloud.

    Tutti i cloud diventano pubblici quando gli ambienti sono suddivisi e ridistribuiti su più \textit{tenant}\footnote{Un tenant può essere un singolo utente ma, più frequentemente, è un gruppo di utenti, come l'organizzazione di un cliente, che condivide un accesso comune all'istanza dell'applicazione e i privilegi al suo interno.\cite{Tenant}}. Nemmeno le strutture tariffarie caratterizzano più i cloud pubblici, poiché alcuni provider consentono ai tenant di usare i loro cloud gratuitamente. L'infrastruttura IT \textit{bare metal,} \footnote{È una forma di cloud service in cui l'utente noleggia una macchina fisica da un provider che non condivide con altri tenant.
    In questo modo gli utenti hanno il controllo completo sulla macchina fisica e hanno la possibilità di scegliere il loro sistema operativo, evitare le problematiche legate ai vicini delle infrastrutture condivise e adattare hardware e software per carichi di lavoro specifici.\cite{BareMetal}} utilizzata dai provider di cloud pubblico, può essere astratta e venduta come IaaS, oppure sviluppata in piattaforma cloud e proposta come PaaS.
    \item Cloud privato: I cloud privati sono ambienti cloud riservati a un singolo utente o gruppo ed eseguiti dietro il loro \textit{firewall.} \footnote{Nell'informatica e nell'ambito delle reti di computer, un firewall è un componente hardware e/o software di difesa perimetrale di una rete, che può anche svolgere funzioni di collegamento tra due o più segmenti di rete, o tra una rete e un computer locale, fornendo dunque una protezione in termini di sicurezza informatica della rete stessa e proteggendo il computer da malware o altri pericoli di internet. \cite{FireWall}}
    Non occorre più che i cloud privati abbiano origine in un'infrastruttura IT on-premise. Le organizzazioni infatti realizzano i propri cloud privati su datacenter in affitto, di proprietà del fornitore e ubicati off-premise.
    
    Quando l'infrastruttura IT sottostante è dedicata a un singolo cliente, con un accesso completamente isolato, tutti i cloud diventano privati.
    Questa strutturazione genera anche diversi sottotipi di cloud privato, tra cui:
    \begin{itemize}
        \item Cloud privato gestito: Permette di creare e usare un cloud privato che viene implementato, configurato e gestito da un fornitore terzo, questo lo rende adatto alle aziende di piccole dimensioni o con personale IT non specializzato.
        \item Cloud dedicato: Si tratta di un cloud contenuto in un altro cloud che può esistere all'interno di un cloud pubblico o di un cloud privato. Ad esempio, il reparto contabilità di un'azienda può utilizzare il proprio cloud dedicato, ubicato all'interno del cloud privato aziendale.
    \end{itemize}
    \item Cloud ibrido: I cloud ibridi appaiono come un singolo ambiente IT creato a partire da più ambienti connessi tramite reti LAN (Local Area Network), WAN (Wide Area Network) o VPN (Virtual Private Network) e/o API.
    Un cloud ibrido può, ad esempio, dover includere:
    \begin{itemize}
        \item Almeno un cloud privato e almeno un cloud pubblico.
        \item Due o più cloud privati.
        \item Due o più cloud pubblici.
        \item Un ambiente bare metal o virtuale connesso almeno a un cloud, pubblico o privato.
    \end{itemize}

    Ogni sistema IT si trasforma in cloud ibrido quando le app che vi risiedono possono muoversi tra più ambienti, separati ma connessi, e questi ultimi devono essere gestiti come un singolo ambiente.
    \item Multicloud: Un ambiente multicloud è formato da più di un servizio cloud e da più di un fornitore di servizi cloud, pubblici o privati. Tutti i cloud ibridi sono multicloud, ma non tutti i multicloud sono cloud ibridi. I multicloud diventano cloud ibridi quando i diversi cloud vengono connessi tramite una forma di integrazione o di orchestrazione.
    L’utilizzo di questo ambiente  è sempre più frequente nelle aziende il cui è obiettivo è migliorare sicurezza e prestazioni attraverso l'utilizzo di più ambienti.
\end{itemize}
\vspace{2cm}
\begin{figure}[!h]
    \centering
    \includegraphics[width=1.0\textwidth]{cloud-architecture-table.jpg}
    \caption{Pro e contro tra cloud pubblico e cloud privato}
    \label{fig:pros and cons of cloud architecture}
    \cite{ProsAndConsOfCloudArchitecture}
\end{figure}


 