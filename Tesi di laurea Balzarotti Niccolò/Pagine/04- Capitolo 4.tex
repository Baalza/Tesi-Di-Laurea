\chapter{Trasformazione Digitale}
La trasformazione digitale utilizza le moderne tecnologie digitali, inclusi tutti i tipi di piattaforme cloud pubbliche, private e ibride, per creare o modificare le esperienze dei clienti, la cultura e i processi aziendali per soddisfare le mutevoli dinamiche di business e di mercato.
La trasformazione digitale è mossa da una serie di tecnologie innovative che stanno rimodellando i settori e ridefinendo le possibilità di business. Queste tecnologie consentono alle organizzazioni di semplificare le operazioni, ottenere informazioni preziose dai dati e offrire prodotti e servizi innovativi. Alcune delle tecnologie principali che guidano la trasformazione digitale includono appunto il cloud computing. \cite{Digitalizzazione}
Ma quali sono i vantaggi e le limitazioni di questa tecnologia?
\section{Vantaggi del cloud computing}
Il passaggio al cloud computing ha trasformato completamente il modo in cui lavoriamo, comunichiamo e collaboriamo, diventando rapidamente una necessità per rimanere competitivi nel mondo digitale di oggi.
Di seguito parleremo dei principali vantaggi e svantaggi e del perché si dovrebbe considerare di passare ai servizi cloud. \cite{ProsAndConsCloud}
\begin{itemize}
    \item Time to market più rapido: 
    puoi avviare nuove istanze o eseguirne il ritiro in pochi secondi, consentendo agli sviluppatori di accelerare lo sviluppo con deployment rapidi. Il cloud computing supporta le nuove innovazioni semplificando l'esecuzione di test di nuove idee e la progettazione di nuove applicazioni senza limitazioni hardware o processi di approvvigionamento lenti.
    \item Scalabilità e flessibilità: 
    Il cloud computing offre maggiore flessibilità alle aziende. Puoi scalare rapidamente le risorse e l'archiviazione per soddisfare le esigenze aziendali senza dover investire in infrastrutture fisiche.
    Le aziende non devono pagare o creare l'infrastruttura necessaria per supportare i loro livelli di carico più elevati. Allo stesso modo, possono fare lo scale down rapidamente se le risorse non vengono utilizzate.  
    \item Risparmi sui costi: 
    Qualunque modello di servizio cloud tu scelga, paghi solo per le risorse effettivamente utilizzate. In questo modo eviterai di sovraccaricare ed effettuare l'overprovisioning del tuo data center e permetterai ai tuoi team IT di risparmiare tempo prezioso da dedicare a un lavoro più strategico. 
    \item Collaborazione più efficace: 
    Il cloud ti consente di rendere i dati disponibili ovunque ti trovi, in qualsiasi momento. Anziché essere vincolati a una località o a un dispositivo specifico, gli utenti possono accedere ai dati da qualsiasi parte del mondo e da qualsiasi dispositivo, purché abbiano una connessione a Internet.
    \item Sicurezza avanzata: 
    Il cloud computing può di fatto rafforzare la tua sicurezza grazie alla profondità e all'ampiezza delle funzionalità di sicurezza, alla manutenzione automatica e alla gestione centralizzata, che implementano misure di sicurezza robuste, tra cui crittografia, gestione delle identità e degli accessi.
    \item Prevenzione della perdita di dati:
    I cloud provider offrono funzionalità di backup e ripristino di emergenza. L'archiviazione dei dati nel cloud anziché localmente può contribuire a prevenire la perdita di dati in caso di emergenza, come malfunzionamenti dell'hardware, minacce dannose o persino semplici errori degli utenti. 
    \item Sostenibilità ambientale: 
    I fornitori di servizi cloud si concentrano sempre di più sulla sostenibilità, incrementando l’efficienza energetica dei data center e scegliendo le fonti rinnovabili. Con i servizi cloud, le organizzazioni possono ridurre le loro emissioni di anidride carbonica e contribuire alla sostenibilità ambientale.\cite{HPECloudArchitecture}
\end{itemize}
\subsection{Limitazioni del cloud computing}
Uno degli svantaggi più comuni del cloud computing è che si basa su una connessione a Internet. L'informatica tradizionale utilizza una connessione cablata per accedere ai dati su server o dispositivi di archiviazione. Con il cloud computing, una connessione errata potrebbe impedirti di accedere alle informazioni o alle applicazioni di cui hai bisogno. 
Anche i principali fornitori di servizi cloud possono riscontrare tempi di inattività a causa di una calamità naturale o di prestazioni più lente causate da un problema tecnico imprevisto che potrebbe influire sulla connettività.
Altre limitazioni potrebbero essere: \cite{ProsAndConsCloud}
\begin{itemize}
    \item Rischio di vincoli al fornitore  
    \item Un minor controllo sull'infrastruttura cloud sottostante
    \item Preoccupazioni relative a rischi per la sicurezza quali privacy dei dati e minacce online
    \item Complessità di integrazione con i sistemi esistenti
    \item Costi e spese imprevisti
\end{itemize}
A questo punto, è evidente che i vantaggi superano le limitazioni. La maggior parte delle aziende oggi non valuta se eseguire la migrazione al cloud, ma cosa dovrebbe migrare. 
Il cloud offre più flessibilità e affidabilità, migliora le prestazioni, l'efficienza e consente di ridurre i costi IT, migliorando inoltre l'innovazione. Questi vantaggi primari possono anche tradursi in altri vantaggi correlati che possono aiutare ad aumentare la produttività, supportare la forza lavoro da remoto e migliorare l'efficienza operativa. 
Inoltre, è importante ricordare che intraprendere il percorso verso il cloud non è necessariamente uno scenario del tipo "o tutto o niente". Ad esempio, molte aziende stanno scoprendo che l'adozione di un approccio ibrido può aiutare ad ampliare la capacità e le funzionalità dell'infrastruttura esistente. 
\section{Migrazione al cloud}
La cosiddetta "Cloud Migration" è al centro del processo di trasformazione digitale delle aziende. Migrare verso il cloud vuol dire sostanzialmente muovere dati, risorse, applicazioni e i vari elementi di business in un ambiente tecnologico condiviso, scalabile e flessibile. Il principale obiettivo è quello di riuscire ad ospitare e valorizzare dati e applicazioni in un ambiente il più ottimale possibile per l’organizzazione, con notevoli benefici in termini di costi, performance e sicurezza.\cite{CloudMigration}
Il cloud computing aiuta le aziende ad archiviare, elaborare e accedere ai dati ed è proprio per questo che abbiamo deciso di utilizzare questa tecnologia nella nostra web app il cui sviluppo e approfondimento saranno oggetto di analisi nei prossimi capitoli.